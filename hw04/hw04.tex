\documentclass[11pt]{article}

% ------
% LAYOUT
% ------
\textwidth 165mm %
\textheight 230mm %
\oddsidemargin 0mm %
\evensidemargin 0mm %
\topmargin -15mm %
\parindent= 10mm

\usepackage[dvips]{graphicx}
\usepackage{multirow,multicol}
\usepackage[table]{xcolor}

\usepackage{amssymb}
\usepackage{amsfonts}
\usepackage{amsthm}
\usepackage{amsmath}

\usepackage{subfigure}
\usepackage{minted}

\graphicspath{{./pix/}} % put all your figures here.

\begin{document}
\begin{center}
\Large{\textbf{ECE 595: Homework 4}}

Yi Qiao, Class ID 187

(Spring 2019)
\end{center}

\section*{Exercise 1: Theory}
\subsection*{(a) Logistic regression}
\subsubsection*{(i)}
$$\underset{\pmb{\theta}}{argmin} \sum_{j=1}^{N}-\left\{y_j\log h_{\pmb{\theta}}(\pmb{x}_j)+\left(1-y_j\right) \log (1-h_{\pmb{\theta}}(\pmb{x}_j)) \right\}$$

\begin{equation}
\begin{split}
&=\underset{\pmb{\theta}}{argmin} \sum_{j=1}^{N}-\left\{\log (1-h_{\pmb{\theta}}(\pmb{x}_j))+y_j \log \frac{h_{\pmb{\theta}}(\pmb{x}_j)}{1-h_{\pmb{\theta}}(\pmb{x}_j)} \right\}\\
&=\underset{\pmb{\theta}}{argmin} \sum_{j=1}^{N}-\left\{\log (1-h_{\pmb{\theta}}(\pmb{x}_j))+y_j \log \frac{\frac{1}{1+e^{-\pmb{\theta}^T\pmb{x}_j}}}{\frac{e^{-\pmb{\theta}^T\pmb{x}_j}}{1+e^{-\pmb{\theta}^T\pmb{x}_j}}} \right\}\\
&=\underset{\pmb{\theta}}{argmin} \sum_{j=1}^{N}-\left\{\log(1-h_{\pmb{\theta}}(\pmb{x}_j))+y_j\pmb{\theta}^T\pmb{x}_j \right\}
\end{split}
\end{equation}
Take the derivative..
\begin{equation}
\begin{split}
\nabla_{\pmb{\theta}}\sum_{j=1}^{N}-\left\{\log(1-h_{\pmb{\theta}}(\pmb{x}_j))+y_j\pmb{\theta}^T\pmb{x}_j \right\}&=0\\
\sum_{j=1}^{N}-\left\{\nabla_{\pmb{\theta}}\log(1-h_{\pmb{\theta}}(\pmb{x}_j))+\nabla_{\pmb{\theta}}y_j\pmb{\theta}^T\pmb{x}_j \right\}&=0\\
\sum_{j=1}^{N}\left\{\pmb{x}_jh_{\pmb{\theta}}(\pmb{x}_j)-y_j\pmb{x}_j \right\}&=0\\
\sum_{j=1}^{N}\left\{\left(h_{\pmb{\theta}}(\pmb{x}_j)-y_j\right)\pmb{x}_j \right\}&=0\\
h_{\pmb{\theta}}(\pmb{x}_j)&=y_j
\end{split}
\end{equation}
Which implies, when linearly separable...\\
When $y_j = 1$,
$$\pmb{\theta}^T\pmb{x}_j = \infty$$
When $y_j = 0$,
$$\pmb{\theta}^T\pmb{x}_j = -\infty$$
Since $x_j$ is always bounded, obviously, $\pmb{\theta}$ tends to $\infty$. 
\pagebreak

\noindent Since $\pmb{\theta}$ goes to $\infty$ for $\nabla_{\pmb{\theta}}$ goes to $0$, for any finite $\pmb{\theta}$, the derivative 
$$ \sum_{j=1}^{N}\left\{\left(h_{\pmb{\theta}}(\pmb{x}_j)-y_j\right)\pmb{x}_j \right\}$$
will never goes to $0$. Thus the gradient decent,
$$\pmb{\theta}^{(k+1)}=\pmb{\theta}^{(k)}-\alpha_k\left(\sum_{j=1}^{N}\left(h_{\pmb{\theta}^{(k)}}(\pmb{x}_j)-y_j\right)\pmb{x}_j\right)$$
will never stop for any finite $k$.

\subsubsection*{(ii)}
If linearly separable, the discriminant function will go as steep as possible. Thus, $\pmb{w},\ w_0$ will go as far as possible.\\
We can also add some regularization to the original optimization problem, like
$$\underset{\pmb{\theta}}{argmin} \sum_{j=1}^{N}-\left\{y_j\log h_{\pmb{\theta}}(\pmb{x}_j)+\left(1-y_j\right) \log (1-h_{\pmb{\theta}}(\pmb{x}_j)) \right\}+\lambda||\pmb{\theta}||^2$$

\subsubsection*{(iii)}
No. Since no one else use $\log$ in the loss function.

\subsection*{(b) Perceptron}

\subsubsection*{(i)}

$$\pmb{w}^{(k+1)}=\pmb{w}^{(k)}+y_j\pmb{x}_j,\ some\ j\in \mathcal{M}_k$$

\begin{equation}
\begin{split}
y_j(\pmb{w}^{(k+1)})^T\pmb{x}_j &= y_j(\pmb{w}^{(k)}+y_j\pmb{x}_j)^T\pmb{x}_j\\
&= y_j(\pmb{w}^{(k)})^T\pmb{x}_j+y_j^2\pmb{x}_j^T\pmb{x}_j\\
&= y_j(\pmb{w}^{(k)})^T\pmb{x}_j+y_j^2||\pmb{x}_j||^2_2
\end{split}
\end{equation}

The second term $y_j^2||\pmb{x}_j||^2_2 \ge0$, thus,
$$y_j(\pmb{w}^{(k+1)})^T\pmb{x}_j>y_j(\pmb{w}^{(k)})^T\pmb{x}_j$$

Thus making the loss function smaller.
\pagebreak
\subsubsection*{(ii)} 
\noindent \textbf{Step 1.} Show that $(\pmb{w}^{(k)})^T\pmb{w}^*\ge k\rho$\\
Assuming $\pmb{w}^{(0)} = 0$, obviously, $(\pmb{w}^{(0)})^T\pmb{w}^* = 0 \ge 0\rho$\\
Assuming $(\pmb{w}^{(k-1)})^T\pmb{w}^*\ge (k-1)\rho$,
\begin{equation}
\begin{split}
(\pmb{w}^{(k)})^T\pmb{w}^* &= (\pmb{w}^{(k-1)} + y_j\pmb{x}_j)^T\pmb{w}^*,\ some\ j\in  \mathcal{M}_k\\
&=(\pmb{w}^{(k-1)})^T\pmb{w}^*+y_j\pmb{x}_j^T\pmb{w}^*\\
&\ge (\pmb{w}^{(k-1)})^T\pmb{w}^*+\min_j y_j\pmb{x}_j^T\pmb{w}^*\\
&\ge (\pmb{w}^{(k-1)})^T\pmb{w}^*+\rho\\
&\ge k\rho
\end{split}
\end{equation}

\noindent By induction, $(\pmb{w}^{(k)})^T\pmb{w}^*\ge k\rho$ is proved.\\
\noindent \textbf{Step 2.} Show that $||\pmb{w}^{(k)}||^2_2 \le ||\pmb{w}^{(k-1)}||^2_2+||\pmb{x}_j||^2_2$

\begin{equation}
\begin{split}
||\pmb{w}^{(k)}||^2_2 &= ||\pmb{w}^{(k-1)} + y_j\pmb{x}_j||^2_2,\ some\ j\in\mathcal{M}_{k-1}\\
&=||\pmb{w}^{(k-1)}||^2_2+y_j^2||\pmb{x}_j||^2_2+2y_j(\pmb{w}^{(k-1)})^Tx_j1,\ some\ j\in\mathcal{M}_{k-1}
\end{split}
\end{equation}
Since $j \in \mathcal{M}_{k-1}$, $y_j(\pmb{w}^{(k-1)})^T\pmb{x}_j < 0$, thus,
$$||\pmb{w}^{(k)}||^2_2 \le ||\pmb{w}^{(k-1)}||^2_2+||\pmb{x}_j||^2_2$$
is proved.\\
Also, 
\begin{equation}
\begin{split}
||\pmb{w}^{(k)}||^2_2 &\le ||\pmb{w}^{(k-1)}||^2_2+\max_j||\pmb{x}_j||^2_2\\
&=||\pmb{w}^{(k-1)}||^2_2+R^2
\end{split}
\end{equation}
By induction, we get,
$$||\pmb{w}^{(k)}||^2_2 \le kR^2$$

\noindent \textbf{Step 3.}\\
Using 1. and 2., by inspection,
\begin{equation}
\begin{split}
\frac{(\pmb{w}^{(k)})^T\pmb{w}^*}{||\pmb{w}^{(k)}||_2}&\ge \frac{k\rho}{\sqrt{kR^2}}\\
&= \sqrt{k}\frac{\rho}{R}
\end{split}
\end{equation}
Which implies,
\begin{equation}
\begin{split}
\sqrt{k}&\le \frac{R(\pmb{w}^{(k)})^T\pmb{w}^*}{\rho||\pmb{w}^{(k)}||_2}\\
k&\le \frac{R^2(\pmb{w}^{(k)})^T(\pmb{w}^{(k)})\pmb{w}^{*T}\pmb{w}^*}{\rho^2||\pmb{w}^{(k)}||^2_2}\\
&\le \frac{R^2||\pmb{w}^*||^2_2}{\rho^2}
\end{split}
\end{equation}

\pagebreak
\subsection*{(c) SVM}
\subsubsection*{(i)}
$$\underset{\pmb{\theta},\pmb{\xi}}{argmin}\ \frac{1}{2}||\pmb{w}||^2_2+C||\pmb{\xi}||^2_2$$
$$subject\ to\ y_jg_{\pmb{\theta}}(\pmb{x}_j)\ge 1-\xi_j,\ \xi_j\ge 0\ \forall j$$

\noindent By taking the derivative, we get,
\begin{equation}
\begin{split}
&\nabla_{\pmb{\xi}}\frac{1}{2}||\pmb{w}||^2_2+C||\pmb{\xi}||^2_2=2C\pmb{\xi}\\
&\nabla_{\pmb{\theta}}\frac{1}{2}||\pmb{w}||^2_2+C||\pmb{\xi}||^2_2=\pmb{w}
\end{split} 
\end{equation}

\noindent When $2C\pmb{\xi}=0$, obviously $\xi_j \ge 0$ is satisfied.\\
When $\pmb{w}=0$, $0\ge 1-\xi_j$, which implies $\xi_j\ge 1$.\\
No matter bounded by which one, $\xi_j \ge 0$ is satisfied, thus such condition has no effect on the solution.
\subsubsection*{(ii)}
\begin{equation}
\begin{split}
&\underset{\pmb{\theta},\pmb{\xi}}{argmin}\ \frac{1}{2}||\pmb{w}||^2_2 + \frac{1}{2}C||\pmb{\xi}||^2_2\\
&subject\ to\ y_jg_{\pmb{\theta}}(\pmb{x}_j)\ge 1-\xi_j,j=1,..,N
\end{split}
\end{equation}
Writing out the Lagrangian, 
\begin{equation}
\begin{split}
\mathcal{L}(\pmb{\theta}, \pmb{\xi}, \pmb{\mu}) &= \frac{1}{2}||\pmb{w}||^2_2 + \frac{1}{2}C||\pmb{\xi}||^2_2 - \sum_{j=1}^{N}\mu_j(\xi_j-1+y_jg_{\pmb{\theta}}(\pmb{x}_j))\\
&=\frac{1}{2}\pmb{w}^T\pmb{w} + \frac{1}{2}C\pmb{\xi}^T\pmb{\xi} + \sum_{j=1}^{N}\mu_j(1-y_jw_0) + \sum_{j=1}^{N}-\mu_j\xi_j + \sum_{j=1}^{N}-\mu_jy_j\pmb{w}^T\pmb{x}_j\\
&=(\frac{1}{2}\pmb{w}-\sum_{j=1}^{N}\mu_jy_j\pmb{x}_j)^T\pmb{w} + (\frac{1}{2}C\pmb{\xi}-\pmb{\mu})^T\pmb{\xi} + \sum_{j=1}^{N}\mu_j(1-y_jw_0)\\
\nabla_{\pmb{\theta}}\mathcal{L}(\pmb{w}, \pmb{\xi}, \pmb{\mu}) &= \pmb{w}-\sum_{j=1}^{N}\mu_jy_j\pmb{x}_j = 0\\
\nabla_{w_0}\mathcal{L}(\pmb{\theta}, \pmb{\xi}, \pmb{\mu}) &=\sum_{j=1}^{N}-\mu_j y_j=0\\
\nabla_{\pmb{\xi}}\mathcal{L}(\pmb{\theta}, \pmb{\xi}, \pmb{\mu}) &= C\pmb{\xi}-\pmb{\mu} = 0\\
\end{split}
\end{equation}
Thus,
\begin{equation}
\begin{split}
&\pmb{w}^*=\sum_{j=1}^{N}\mu_jy_j\pmb{x}_j\\
&\sum_{j=1}^{N}\mu_jy_j = 0\\
&C\xi_j^*= \mu_j\ \forall j\\
\end{split}
\end{equation}

\subsubsection*{(iii)}
\begin{equation}
\begin{split}
\mathcal{L}(\pmb{\theta}, \pmb{\xi}, \pmb{\mu}) &= \frac{1}{2}||\pmb{w}||^2_2 + \frac{1}{2}C||\pmb{\xi}||^2_2 - \sum_{j=1}^{N}\mu_j(\xi_j-1+y_jg_{\pmb{\theta}}(\pmb{x}_j))\\
\end{split}
\end{equation}
Complementarity Conditions says,
\begin{equation}
\begin{split}
\mu_j > 0\ and\ \xi_j-1+y_jg_{\pmb{\theta}}(\pmb{x}_j) = 0\\
\mu_j = 0\ and\ \xi_j-1+y_jg_{\pmb{\theta}}(\pmb{x}_j) > 0\\
\end{split}
\end{equation} 
Which means,
\begin{equation}
\nabla_{\pmb{\mu}}\mathcal{L}(\pmb{\theta}, \pmb{\xi}, \pmb{\mu})=\pmb{0} \Rightarrow \underset{\pmb{\mu}\ge\pmb{0}}{argmax}\left\{- \sum_{j=1}^{N}\mu_j(\xi_j-1+y_jg_{\pmb{\theta}}(\pmb{x}_j))\right\}
\end{equation}
Then, the primal problem becomes,
\begin{equation}
\begin{split}
\underset{\pmb{\theta},\pmb{\xi}}{argmin}\ \mathcal{L}(\pmb{\theta}, \pmb{\xi}, \pmb{\mu}^*) &= \underset{\pmb{\theta},\pmb{\xi}}{argmin}\left\{ \frac{1}{2}||\pmb{w}||^2_2 + \frac{1}{2}C||\pmb{\xi}||^2_2 + \underset{\pmb{\mu}\ge\pmb{0}}{argmax}\left\{- \sum_{j=1}^{N}\mu_j(\xi_j-1+y_jg_{\pmb{\theta}}(\pmb{x}_j))\right\}\right\}\\
&=\min_{\pmb{\theta},\pmb{\xi}}\max_{\pmb{\mu}\ge\pmb{0}}\mathcal{L}(\pmb{\theta}, \pmb{\xi}, \pmb{\mu})
\end{split}
\end{equation}
The dual problem is,
\begin{equation}
\begin{split}
\max_{\pmb{\mu}\ge\pmb{0}}\min_{\pmb{\theta},\pmb{\xi}}\mathcal{L}(\pmb{\theta}, \pmb{\xi}, \pmb{\mu})&= \max_{\pmb{\mu}\ge\pmb{0}}\left\{\min_{\pmb{\theta},\pmb{\xi}}\frac{1}{2}||\pmb{w}||^2_2 + \frac{1}{2}C||\pmb{\xi}||^2_2 - \sum_{j=1}^{N}\mu_j(\xi_j-1+y_jg_{\pmb{\theta}}(\pmb{x}_j))\right\}
\end{split}
\end{equation}

Plug in solution we got in (ii),

\begin{equation}
\begin{split}
&=\max_{\pmb{\mu}\ge\pmb{0}}\left\{\frac{1}{2}||\sum_{j=1}^{N}\mu_jy_j\pmb{x}_j||^2_2 + \frac{1}{2}C||\frac{\pmb{\mu}}{C}||^2_2 - \sum_{j=1}^{N}\mu_j(\frac{\mu_j}{C}-1+y_j(\sum_{j=1}^{N}\mu_jy_j\pmb{x}_j^T\pmb{x}_j+w_0))\right\}\\
&=\max_{\pmb{\mu}\ge\pmb{0}}\left\{\frac{1}{2}\sum_{i=1}^{N}\sum_{j=1}^{N}\mu_i\mu_jy_iy_j\pmb{x}_i^T\pmb{x}_j + \frac{1}{2C}\pmb{\mu}^T\pmb{\mu} - \sum_{j=1}^{N}\frac{\mu^2_j}{C}+\sum_{j=1}^{N}\mu_j-\sum_{j=1}^{N}\mu_jy_j(\sum_{j=1}^{N}\mu_jy_j\pmb{x}_j^T\pmb{x}_j+w_0))\right\}\\
&=\max_{\pmb{\mu}\ge\pmb{0}}\left\{\sum_{j=1}^{N}\mu_j+\frac{1}{2}\sum_{i=1}^{N}\sum_{j=1}^{N}\mu_i\mu_jy_iy_j\pmb{x}_i^T\pmb{x}_j-\frac{1}{2}\sum_{j=1}^{N}\frac{\mu^2_j}{C}\right\}\ subject\ to\ \sum_{j=1}^{N}\mu_jy_j = 0\\
\end{split}
\end{equation}
\pagebreak
\subsection*{(d) SVM}
\subsubsection*{(i)}
\begin{equation}
\begin{split}
&\underset{\pmb{\theta}}{argmin}\ \frac{1}{2}||\pmb{w}||^2_2 \\
&subject\ to\ y_jg_{\pmb{\theta}}(\pmb{x}_j)\ge \gamma,j=1,...,N
\end{split}
\end{equation}

Scaling the subjective function,
$$y_j(\frac{\pmb{w}^T}{\gamma}\pmb{x}_j+\frac{w_0}{\gamma}) \ge 1$$

Substituting $\frac{\pmb{w}^T}{\gamma}\rightarrow \pmb{w}^T$, $\frac{w_0}{\gamma}\rightarrow w_0$, The problem becomes

\begin{equation}
\begin{split}
&\underset{\gamma\pmb{\theta}}{argmin}\ \frac{\gamma^2}{2}||\pmb{w}||^2_2 \\
&subject\ to\ y_jg_{\pmb{\theta}}(\pmb{x}_j)\ge 1,j=1,...,N
\end{split}
\end{equation}

which is the same as,

\begin{equation}
\begin{split}
&\underset{\gamma\pmb{\theta}}{argmin}\ \frac{1}{2}||\pmb{w}||^2_2 \\
&subject\ to\ y_jg_{\pmb{\theta}}(\pmb{x}_j)\ge 1,j=1,...,N
\end{split}
\end{equation}

By inspection, the solution will be 
$$\pmb{\theta}^* = \frac{1}{\gamma}\pmb{\theta}^*_{original}$$

Since the decision boundary is,
$$\pmb{w}^T\pmb{x} + w_0=0$$

multiply $\pmb{\theta}$ by some scalar $\frac{1}{\gamma}$, the new decision boundary becomes,
$$\frac{1}{\gamma}\pmb{w}^T\pmb{x} + \frac{1}{\gamma} w_0 = 0$$

which is the same as the original decision boundary.
\pagebreak
\subsubsection*{(ii)}
Suppose we have two data points, $\{\pmb{x}_0, -1\}$ and $\{\pmb{x}_1, 1\}$,
the Hard-Margin SVM problem is:
\begin{equation}
\begin{split}
&\underset{\pmb{\theta}}{argmin}\ \frac{1}{2}||\pmb{w}||^2_2\\
&subject\ to\ -(\pmb{w}^T\pmb{x}_0+w_0) \ge 1,\ \pmb{w}^T\pmb{x}_1+w_0 \ge 1
\end{split}
\end{equation}
Writing out the Lagrangian,

\begin{equation}
\begin{split}
\mathcal{L}(\pmb{w},w_0,\pmb{\mu}) &= \frac{1}{2}||\pmb{w}||^2_2 - \mu_0(-(\pmb{w}^T\pmb{x}_0+w_0)-1) - \mu_1(\pmb{w}^T\pmb{x}_1+w_0-1)\\
\nabla_{\pmb{w}}\mathcal{L}(\pmb{w},w_0,\pmb{\mu}) &= \pmb{w} + \mu_0\pmb{x}_0 - \mu_1\pmb{x}_1 = 0\\
\nabla_{w_0}\mathcal{L}(\pmb{w},w_0,\pmb{\mu}) &= \mu_0 - \mu_1 = 0
\end{split}
\end{equation}
The primal problem is,

\begin{equation}
\begin{split}
\underset{\pmb{w},w_0}{argmin}\ \mathcal{L}(\pmb{w},w_0,\pmb{\mu}^*) &= \min_{\pmb{w}, w_0} \left\{\frac{1}{2}||\pmb{w}||^2_2 + \max_{\pmb{\mu}\ge\pmb{0}}\left\{- \mu_0(-(\pmb{w}^T\pmb{x}_0+w_0)-1) - \mu_1(\pmb{w}^T\pmb{x}_1+w_0-1)\right\}\right\}\\
\end{split}
\end{equation} 
The dual problem is,
\begin{equation}
\begin{split}
&\max_{\pmb{\mu}\ge\pmb{0}}\min_{\pmb{w}, w_0} \left\{\frac{1}{2}||\pmb{w}||^2_2 +\left\{- \mu_0(-(\pmb{w}^T\pmb{x}_0+w_0)-1) - \mu_1(\pmb{w}^T\pmb{x}_1+w_0-1)\right\}\right\}\\
&=\max_{\pmb{\mu}\ge\pmb{0}}\min_{\pmb{w}, w_0} \left\{\frac{1}{2}||\mu_1\pmb{x}_1-\mu_0\pmb{x}_0||^2_2+\mu_0(\pmb{w}^T\pmb{x}_0+w_0)+\mu_0 - \mu_1(\pmb{w}^T\pmb{x}_1+w_0)+\mu_1\right\}\\
&=\max_{\pmb{\mu}\ge\pmb{0}} \left\{\frac{1}{2}\mu_0^2(\pmb{x}_1-\pmb{x}_0)^T(\pmb{x}_1-\pmb{x}_0)-\mu_0^2(\pmb{x}_1-\pmb{x}_0)^T(\pmb{x}_1-\pmb{x}_0)+2\mu_0 \right\}\\
&=\max_{\pmb{\mu}\ge\pmb{0}} \left\{-\frac{1}{2}\mu_0^2(\pmb{x}_1-\pmb{x}_0)^T(\pmb{x}_1-\pmb{x}_0)+2\mu_0 \right\}\\
\end{split}
\end{equation}
Taking derivative,
\begin{equation}
\begin{split}
&\frac{\partial }{\partial \mu_0}\left\{-\frac{1}{2}\mu_0^2(\pmb{x}_1-\pmb{x}_0)^T(\pmb{x}_1-\pmb{x}_0)+2\mu_0 \right\}\\
&=-\mu_0(\pmb{x}_1-\pmb{x}_0)^T(\pmb{x}_1-\pmb{x}_0)+2 = 0 
\end{split}
\end{equation}
which implies,
\begin{equation}
\begin{split}
&\mu_0^* = \mu_1^* = \frac{2}{(\pmb{x}_1-\pmb{x}_0)^T(\pmb{x}_1-\pmb{x}_0)}\\
&\pmb{w}^* = \mu_0(\pmb{x}_1-\pmb{x}_0) = \frac{2}{(\pmb{x}_1-\pmb{x}_0)^T(\pmb{x}_1-\pmb{x}_0)} (\pmb{x}_1-\pmb{x}_0)\\
&w_0^* = -\frac{(\pmb{x}_0 + \pmb{x}_1)^T\pmb{w}^*}{2} = -\frac{(\pmb{x}_0 + \pmb{x}_1)^T(\pmb{x}_1-\pmb{x}_0)}{(\pmb{x}_1-\pmb{x}_0)^T(\pmb{x}_1-\pmb{x}_0)}
\end{split}
\end{equation}
\end{document}